\documentclass{article}

% if you need to pass options to natbib, use, e.g.:
% \PassOptionsToPackage{numbers, compress}{natbib}
% before loading nips_2016
%
% to avoid loading the natbib package, add option nonatbib:
% \usepackage[nonatbib]{nips_2016}

\usepackage[final]{nips_2016}

% to compile a camera-ready version, add the [final] option, e.g.:
% \usepackage[final]{nips_2016}

\usepackage[utf8]{inputenc} % allow utf-8 input
\usepackage[T1]{fontenc}    % use 8-bit T1 fonts
\usepackage{hyperref}       % hyperlinks
\usepackage{url}            % simple URL typesetting
\usepackage{booktabs}       % professional-quality tables
\usepackage{amsfonts}       % blackboard math symbols
\usepackage{nicefrac}       % compact symbols for 1/2, etc.
\usepackage{microtype}      % microtypography

\title{The Spectral Density Kernel and Portfolio Optimization}

\author{
  Nathaniel L. Diamant \\
  Department of Mathematics\\
  Harvey Mudd College\\
  \texttt{ndiamant@hmc.edu} \\
  \And Kyle Suver \\
  Department of Mathematics\\
  Harvey Mudd College\\
  \texttt{ksuver@hmc.edu} \\
}

\begin{document}

\maketitle

\begin{abstract}
  Gaussian processes allow smooth data interpolation and regression with confidence bounds, but until Andrew Gordon Wilson and Ryan Prescott Adams' research [1], often required human intuition to find the correct Kernel. We apply their Spectral Density Kernel to stock data of different resolutions along with simple portfolio optimization techniques to explore a novel trading strategy. We find...
\end{abstract}

\section{Introduction}
\label{intro}

Discuss GPs and spectral density in vague terms and introduce our goals.

\subsection{Gaussian Processes (GPs)}

\subsection{Spectral Mixture Kernel}

\subsection{Portfolio Optimization}

\subsection{Portfolio optimization}

\section{Background}
\label{background}

Here we introduce the math of GPs and the spectral density kernel

\section{Problem formulation}
\label{problem}

\subsection{GP application}

Here we explain how we use the spectral density kernel and our optimization strategy.

\subsection{Optimization}

\section{Experiments}
\label{exper}

The results of our optimization.

\subsection{Citations within the text}

\subsection{Figures}

\begin{figure}[h]
  \centering
  \fbox{\rule[-.5cm]{0cm}{4cm} \rule[-.5cm]{4cm}{0cm}}
  \caption{Sample figure caption.}
\end{figure}

\subsection{Tables}

Note that publication-quality tables \emph{do not contain vertical
  rules.} We strongly suggest the use of the \verb+booktabs+ package,
which allows for typesetting high-quality, professional tables:
\begin{center}
  \url{https://www.ctan.org/pkg/booktabs}
\end{center}
This package was used to typeset Table~\ref{sample-table}.

\begin{table}[t]
  \caption{Sample table title}
  \label{sample-table}
  \centering
  \begin{tabular}{lll}
    \toprule
    \multicolumn{2}{c}{Part}                   \\
    \cmidrule{1-2}
    Name     & Description     & Size ($\mu$m) \\
    \midrule
    Dendrite & Input terminal  & $\sim$100     \\
    Axon     & Output terminal & $\sim$10      \\
    Soma     & Cell body       & up to $10^6$  \\
    \bottomrule
  \end{tabular}
\end{table}

\section{Discussion}
\label{discuss}

Describe results and potential future work.

\begin{itemize}

\item The \verb+\bbold+ package almost always uses bitmap fonts.  You
  should use the equivalent AMS Fonts:
\begin{verbatim}
   \usepackage{amsfonts}
\end{verbatim}
followed by, e.g., \verb+\mathbb{R}+, \verb+\mathbb{N}+, or
\verb+\mathbb{C}+ for $\mathbb{R}$, $\mathbb{N}$ or $\mathbb{C}$.  You
can also use the following workaround for reals, natural and complex:
\begin{verbatim}
   \newcommand{\RR}{I\!\!R} %real numbers
   \newcommand{\Nat}{I\!\!N} %natural numbers
   \newcommand{\CC}{I\!\!\!\!C} %complex numbers
\end{verbatim}

\end{itemize}

\cite{spectral}

\bibliographystyle{unsrt}
\bibliography{bibliography}

\end{document}